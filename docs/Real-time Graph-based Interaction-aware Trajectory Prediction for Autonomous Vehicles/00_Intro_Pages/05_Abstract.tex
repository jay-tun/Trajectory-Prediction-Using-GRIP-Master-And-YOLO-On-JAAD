%~~~~~~~~~~~~~~~~~~~~~~~~~~~~~~~~~~~~~~~~~~~~~~~~~~~~~~~~~~~~~~~~~~~~~~~~~~~~~~~~~~~~~~~~~%
%	                                     ABSTRACT   	                                  %
%~~~~~~~~~~~~~~~~~~~~~~~~~~~~~~~~~~~~~~~~~~~~~~~~~~~~~~~~~~~~~~~~~~~~~~~~~~~~~~~~~~~~~~~~~%
\vspace{0.5pt}
\begin{flushleft}
    \setlength{\parskip}{0pt}
        %\bigskip
    {\raggedright{{\Large{\bf{ABSTRACT}}}} \par}
    \bigskip
    \vspace{6pt}
    

\end{flushleft} % This section is not essential for the abstract

As autonomous vehicles are entering the market more rapidly with advancements in Machine Learning and Computer Vision, it becomes a necessity to develop a robust algorithm that allows the vehicles to predict the trajectory motion of their nearby objects. One such established algorithm, comparable to the state-of-the-art method, is (Enhanced) Graph-based Interaction-aware Trajectory Prediction(GRIP++). The algorithm uses Gated Recurrent Units (GRU) and Long Short-term Memory (LSTM) type of Recurrent Neural Network(RNN) to train the model with the ApolloScape 3D Trajectory dataset. This paper studies the algorithm in detail, implements object detection and semantic segmentation, and evaluates the results using the Joint Attention in Autonomous Driving (JAAD) dataset, which provides data on pedestrian behavior in traffic scenarios.

\vspace{3cm} % Reduce if text overflowing to a new page. Don't make it too long.

\raggedright{{\large \textbf{Keywords: }}}\par{\large Motion Prediction, Trajectory Prediction, Computer Vision, Machine Learning, Deep Learning, Long short-term memory, Recurrent neural networks, Graph-based motion prediction}
\vfill
\clearpage