\chapter{Introduction} 

\tab Development in the area of Artificial Intelligence, Machine Learning, and Computer vision has been surging in the recent couple of years \cite{8793868}. With it, autonomous vehicles are becoming more available commercially \cite{2019itsc_grip}.  Despite advancement in the field, mass production of self-driving cars and their deployment in everyday life would be only possible if their safety is verified \cite{9756903}. This poses a challenge, especially due to the reported traffic incidents in 2018 from leading companies in the area such as Tesla and Uber \cite{2019itsc_grip}, and has caused concern for the safety of autonomous vehicles (AV). 

\tab One of the aspects that need to be improved in order to ensure the safety of AV is predicting the future positions of the surrounding close-by objects \cite{9756903} just like in the case of human drivers. It is argued that if AV are to predict the future locations of the nearby traffic participants precisely, traffic accidents can be avoided \cite{2019itsc_grip}. Therefore, to ensure their safety, it is imperative to improve the performance of the algorithms used in AV for trajectory prediction.

\tab Various algorithms have been developed to that effect, with numerous car manufacturers working towards improving vehicle safety via better designs of Advanced Driver-Assistance Systems (ADAS) \cite{8793868}. However, much needs to be done in the area. Developing technologies for self-driving cars is also not a recent phenomenon. We can trace its development back to as early as the 1980s \cite{8793868}, especially with Dean A. Pomerleau's work on ALVINN in 1989 that utilized neural networks for the purpose of autonomous navigation \cite{Pomerleau-1989-15721}. These early advancements helped us reach where we currently stand today, which offers different algorithms and models tailored for trajectory prediction that deploy different techniques such as the Kalman Filter, Hidden Markov Model, Bayesian Networks, Gaussian Processes, or using Machine Learning models with Convolutional Neural Network (CNN) and Recurrent Neural Network(RNN); more specifically Long Short-Term Memory(LSTM), Gated Recurrent Unit (GRU), Conditional Variational Auto-Encoder (CVAE), etc. \cite{8793868}.

\tab The focus of this paper is on one particular scheme, developed by Li et al. \cite{li2020gripplus}, called (Enhanced) Graph-based Interaction-aware Trajectory Prediction (GRIP++) that uses graph convolutions to extract features and applies LSTM to make predictions. In this research, the GRIP++ algorithm was analyzed and applied to the JAAD dataset \cite{rasouli2017ICCVW} focused on pedestrian behavior in traffic scenarios. To enhance the analysis, object detection was performed using YOLO \cite{yolov3}, identifying traffic agents like pedestrians and vehicles. Additionally, semantic segmentation was applied to the video data using DeepLabV3+ \cite{DBLP:journals/corr/abs-1802-02611}, which allowed for the identification of drivable areas, crosswalks, and other relevant scene elements. The predicted trajectories, obtained after training and testing the GRIP++ model, were then overlaid onto the processed video, providing visualization of how well the model predicts pedestrian movement in a given traffic scene.





